
\subsection{MOY moves for rational $\sl_n$}
The following explicit instances of the lemma should be thought of as refinements of MOY moves.

Circle removal:\\

    \begin{minipage}[c]{0.5\textwidth}
        \begin{align*}
            R &= \Q[\G, x_0-x_1] \\
            M &= R[x_3-x_0] \\
            \overline{W} &= 0 \\
            (n+1)(x_3-x_0)^n-p &= (2x_0-x_1-x_3)Z \\
            q &= 0
        \end{align*}
    \end{minipage}
    \hfill
    \begin{minipage}[c]{0.45\textwidth}
        \[\begin{tikzcd}[column sep=3cm]
            \bigoplus_{i=0}^{n-1}R\ar[dd, shift right=0.6ex, swap]\ar[dd, leftarrow, shift left=0.6ex]&\\\\
            R[x_3-x_0]\arrow[r, "0", shift left=0.5ex, bend left=15]& R[x_3-x_0]\ar[l, "H"]\ar[l, "(2x_0-x_1-x_3)Z", shift left=0.5ex, bend left=15]
        \end{tikzcd}\]
    \end{minipage}

    \vspace{1cm} % Adds some vertical space between the items
Loop removal:\\

    \begin{minipage}[c]{0.5\textwidth}
        \begin{align*}
            R&= \Q[\G, x_0-x_1] \\
            M &= R[x_3-x_0]\\
            \overline{W} &= 0 \\
            (n+1)(x_3-x_0)^{n-1}-p &= Z \\
            q &= 0
        \end{align*}
    \end{minipage}
    \hfill
    \begin{minipage}[c]{0.45\textwidth}
        \[\begin{tikzcd}[column sep=3cm]
            &\bigoplus_{i=0}^{n-2}R\ar[dd, shift right=0.6ex, swap]\ar[dd, leftarrow, shift left=0.6ex]\\\\
            R[x_3-x_0]\arrow[r, "Z", shift left=0.5ex, bend left=15]& R[x_3-x_0]\ar[l, "H"]\ar[l, "0", shift left=0.5ex, bend left=15]
        \end{tikzcd}\]
    \end{minipage}

Digon removal:\\
        \begin{align*}
            R&= \Q[\G, x_0-x_1,x_2-x_1,x_3-x_2] \\
            M &= K(Z(x_0-x_1,x-x_1,x_3-x),(x_0-x)(x_0+x-x_1-x_3))_{R[x-x_1]}\\
            \tilde M &= K(Z(x_0-x_1,x_2-x_1,x_3-x_2),(x_0-x_2)(x_0+x_2-x_1-x_3))_{R}\\
            \overline{W} &= P(x_0-x_1)+P(x_3-x_0)-P(x_2-x_1)-P(x_3-x_2) \\
            (x-x_1)^{2}-p &=  (x-x_2)(x+x_2-x_1-x_3)\\
            q &= Z(x-x_1,x_2-x_1,x_3-x_2)
        \end{align*}
       $$\begin{tikzcd}[column sep=3cm]
        \tilde M\oplus \tilde M\ar[dd, shift right=0.6ex, swap]\ar[dd, leftarrow, shift left=0.6ex]&\\\\
            M\ar[r, "{Z(x-x_1, x_2-x_1,x_3-x_2)}", shift left=0.5ex, bend left=15]& M\ar[l, "H"]\ar[l, "(x-x_2)(x+x_2-x_1-x_3)", shift left=0.5ex, bend left=15]
        \end{tikzcd}$$

Square removal:
        \begin{align*}
            R&= \Q[\G, x_0-x_1,x_2-x_1,x_2-x_3] \\
            M &= K(Z(x_0-x_1,x_2-x_1,x-x_2),(x_0-x_2)(x_0+x_2-x_1-x))_{R[x-x_0]}\\
            M_0 &= K()_{R}\\
            M_1 &= K()_{R}\\
            \overline{W} &= P(x_0-x_1)-P(x_0-x_3)-P(x_2-x_1)+P(x_2-x_3) \\
            (n+1)(x-x_0)^{n-1}-p &=  Z(x_2-x_3,x_0-x_3,x-x_0)\\
            q &= (x_2-x_0)(x_2+x_0-x_3-x)
        \end{align*}
       $$\begin{tikzcd}[column sep=3cm]
        &M_0\oplus \bigoplus_{i=0}^{n-3} M_1\ar[dd, shift right=0.6ex, swap]\ar[dd, leftarrow, shift left=0.6ex]\\\\
            M\ar[r, "{Z(x_2-x_3,x_0-x_3,x-x_0)}", shift left=0.5ex, bend left=15]& M\ar[l, "H"]\ar[l, "(x_2-x_0)(x_2+x_0-x_3-x)", shift left=0.5ex, bend left=15]
        \end{tikzcd}$$

For the first Reidemeister move, we start with the following multifactorization over the ring $R=\Q[x_0-x_1]$:
$$\begin{tikzcd}[column sep=large, ]
	R[x_3-x_0]\ar[ddr, out=-60, in=120]&R[x_3-x_0] \ar[l,"(2x_0-x_1-x_3)Z", bend right=10, swap]\arrow[ddl, out=-120, in=60] \\\\
	R[x_3-x_0]\ar[r,"Z", bend right=10, swap]&R[x_3-x_0],
\end{tikzcd}$$
where $Z=Z(x_0-x_1,x_0-x_1,x_3-x_0)$
Here the cubical filtration degree increases downwards. We can apply Lemma \ref{SummandLemma} to obtain a special 0-deformation retract into a multifactorization
$\bigoplus_{i=0}^{n-1} R \to \bigoplus_{i=0}^{n-2}R$, with the differential described by a matrix
$$\left(\begin{array}{cccc} 1  &&&a_1\\&\ddots&&\vdots \\&&1&a_{n-1}
\end{array}\right)$$
where, up to rescaling, the $a_i$ appear as coefficients of lower order terms in $Z$ with respect to the variable $x_3-x_0$. The differential increases filtration degree 1, so the contractible summand $\bigoplus_{i=0}^{n-2} R \xrightarrow{1} \bigoplus_{i=0}^{n-2}R$ can be cancelled out by a 1-nulhomotopy. After this, we are left with a single copy of $R$.


A similar argument applies to reduce
$$\begin{tikzcd}[column sep=large, ]
	R[x_3-x_0]\ar[ddr, out=-60, in=120]\ar[r,"Z", bend left=10]&R[x_3-x_0] \arrow[ddl, out=-120, in=60] \\\\
	R[x_3-x_0]&R[x_3-x_0]\ar[l,"(2x_0-x_1-x_3)Z", bend left=10],
\end{tikzcd}$$
into $R$ by a special 1-deformation retract.